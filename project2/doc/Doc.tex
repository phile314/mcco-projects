\documentclass[12pt, a4paper, twoside]{article}
\usepackage{hyperref}
\usepackage{fullpage}
\usepackage{bussproofs}
% font size could be 10pt (default), 11pt or 12 pt
% paper size coulde be letterpaper (default), legalpaper, executivepaper,
% a4paper, a5paper or b5paper
% side coulde be oneside (default) or twoside 
% columns coulde be onecolumn (default) or twocolumn
% graphics coulde be final (default) or draft 
%
% titlepage coulde be notitlepage (default) or titlepage which 
% makes an extra page for title 
% 
% paper alignment coulde be portrait (default) or landscape 
%
% equations coulde be 
%   default number of the equation on the rigth and equation centered 
%   leqno number on the left and equation centered 
%   fleqn number on the rigth and  equation on the left side
%   
\title{MMCO Project B}
\author{Marco Vassena  \\
    4110161 \\
    \and
    Philipp Hausmann \\
    4003373 \\
    }

\date{\today}

\newcommand{\sproof}{
  \scriptsize
  \begin{center}
  \begin{prooftree}
  \def\defaultHypSeparation{\hskip .1in}
% whats that doing? ->  \def\fCenter{\models}
}

\newcommand{\eproof}{
  \end{prooftree}
  \end{center}
  \normalsize
}

\newcommand{\bcase}[2]{
  \sproof
  \RightLabel{#2}
  \AxiomC{}
  \UnaryInfC{#1}
  \eproof
}

\begin{document}
\maketitle

\section{Formal Specification}
\label{sec:Spec}
This section formally specifies the type rules used for type-checking a t-diagram.
A t-diagram is considered well-typed, if there is a top-level type which can be derived
using the following rules, ill-typed otherwise.

\subsection{Basic values}
\bcase{Program : Program $l_m$ $()$}{program}
\bcase{Platform : Platform $l_m$}{platform}
\bcase{Interpreter : Program $l_m$ (Platform $l_i$)}{interpreter}
\bcase{Compiler : Program $l_m$ (Compiler $l_i$ $l_o$)}{compiler}

\subsection{Compositional Cases}
\sproof
\RightLabel{execute-plat}
\AxiomC{p : Program $l$ $r$}
\AxiomC{d : Platform $l$}
\BinaryInfC{Execute p d : ()}
\eproof

\sproof
\RightLabel{execute-intp}
\AxiomC{p : Program $l_i$ $r$}
\AxiomC{i : Program $l_m$ (Platform $l_i$)}
\BinaryInfC{Execute p i : Program $l_m$ $r$}
\eproof

\sproof
\RightLabel{compile}
\AxiomC{d1 : Program $l_i$ $r$}
\AxiomC{d2 : Program $l_m$ (Compiler $l_i$ $l_o$)}
\BinaryInfC{Compile d1 d2 : Program $l_o$ $r$}
\eproof

\section{Type Checking}
This section explains the design and the implementation of the program \texttt{tc-tdiag}, which 
enforces the type rules formally introduced in \ref{sec:Spec}.
The entry point of the program is \texttt{src/TcTDiag.hs} and does not require any particular
explanation.
The modules concerned with type-checking are contained in \texttt{Type} module:
\begin{itemize}
  \item \texttt{Type.AG}
  \item \texttt{Type.Internal}
  \item \texttt{Type.Error}
\end{itemize}
The module \texttt{Type.AG} contains the attribute grammar that actually type-checks the input 
and produces error messages.
The module \texttt{Type.Internal} defines the data type \texttt{Type}, that encodes t-diagram types.
The implementation replicates exactly the specification introduced in \ref{sec:Spec}.
The module \texttt{Type.Error} contains further sub-modules for producing specific errors 
(\texttt{Type.Error.EType} for type errors and \texttt{Type.Error.EScope} for scoping errors).
Particular effort has been put in making error messages as readable and informative as possible,
taking care of actually hiding the internal representation of types and producing precise source
position references.

\end{document}
