\documentclass[12pt, a4paper, twoside]{article}
\usepackage{hyperref}
\usepackage{fullpage}
\usepackage{bussproofs}
\usepackage{color}
% font size could be 10pt (default), 11pt or 12 pt
% paper size coulde be letterpaper (default), legalpaper, executivepaper,
% a4paper, a5paper or b5paper
% side coulde be oneside (default) or twoside 
% columns coulde be onecolumn (default) or twocolumn
% graphics coulde be final (default) or draft 
%
% titlepage coulde be notitlepage (default) or titlepage which 
% makes an extra page for title 
% 
% paper alignment coulde be portrait (default) or landscape 
%
% equations coulde be 
%   default number of the equation on the rigth and equation centered 
%   leqno number on the left and equation centered 
%   fleqn number on the rigth and  equation on the left side
%   
\title{MMCO Project B}
\author{Marco Vassena  \\
    4110161 \\
    \and
    Philipp Hausmann \\
    4003373 \\
    }

\date{\today}

\newcommand{\sproof}{
  \scriptsize
  \begin{center}
  \begin{prooftree}
  \def\defaultHypSeparation{\hskip .1in}
% whats that doing? ->  \def\fCenter{\models}
}

\newcommand{\eproof}{
  \end{prooftree}
  \end{center}
  \normalsize
}

\newcommand{\bcase}[2]{
  \sproof
  \RightLabel{#2}
  \AxiomC{}
  \UnaryInfC{#1}
  \eproof
}

\newcommand{\blue}[1]{\textcolor{blue}{#1} }
\newcommand{\red}[1]{\textcolor{red}{#1}}

\begin{document}
\maketitle

\section{Formal Specification}
\label{sec:Spec}
This section formally specifies the type rules used for type-checking a t-diagram.
A t-diagram is considered well-typed, if there is a top-level type which can be derived
using the following rules, ill-typed otherwise.
In the following rules \blue{blue} encodes \blue{values} 
(\texttt{Diag\_} as defined in module \texttt{CCO.Diag}) 
and \red{red} encodes t-diagrams \red{types}.

% red for types, blue for constructors

\subsection{Basic values}
\bcase{\blue{Program \_ $l$} : \red{Program $l$ $()$}}{[program]}
\bcase{\blue{Platform $l_m$} : \red{Platform $l_m$}}{[platform]}
\bcase{\blue{Interpreter \_ $l_i$ $l_m$} : \red{Program $l_m$ (Platform $l_i$)}}{[interpreter]}
\bcase{\blue{Compiler \_ $l_1$ $l_2$ $l_m$} : \red{Program $l_m$ (Compiler $l_i$ $l_o$)}}{[compiler]}

\subsection{Compositional Cases}
\sproof
\RightLabel{[execute-on-platform]}
\AxiomC{\blue{d1} : \red{Program $l$ $r$}}
\AxiomC{\blue{d2} : \red{Platform $l$}}
\BinaryInfC{\blue{Execute d1 d2} : \red{$r$}}
\eproof

\sproof
\RightLabel{[execute-on-interpreter]}
\AxiomC{\blue{d1} : \red{Program $l_i$ $r$}}
\AxiomC{\blue{d1} : \red{Program $l_m$ (Platform $l_i$)}}
\BinaryInfC{\blue{Execute d1 d2} : \red{Program $l_m$ $r$}}
\eproof

\sproof
\RightLabel{[compile]}
\AxiomC{\blue{d1} : \red{Program $l_1$ $r$}}
\AxiomC{\blue{d2} : \red{Program $l_m$ (Compiler $l_1$ $l_2$)}}
\BinaryInfC{\blue{Compile d1 d2} : \red{Program $l_2$ $r$}}
\eproof

\section{Type Checking}
This section explains the design and the implementation of the program \texttt{tc-tdiag}, which 
enforces the type rules formally introduced in \ref{sec:Spec}.
The entry point of the program is \texttt{src/TcTDiag.hs} and does not require any particular
explanation.
The modules concerned with type-checking are contained in \texttt{Type} module:
\begin{itemize}
  \item \texttt{Type.AG}
  \item \texttt{Type.Internal}
  \item \texttt{Type.Error}
\end{itemize}
The module \texttt{Type.AG} contains the attribute grammar that actually type-checks the input 
and produces error messages.
The module \texttt{Type.Internal} defines the data type \texttt{Type}, that encodes t-diagram types.
The implementation replicates exactly the specification introduced in \ref{sec:Spec}.
The module \texttt{Type.Error} contains further sub-modules for producing specific errors 
(\texttt{Type.Error.EType} for type errors and \texttt{Type.Error.EScope} for scoping errors).
Particular effort has been put in making error messages as readable and informative as possible,
taking care of hiding the internal representation of types and producing precise source
position references.
This phase has been tested extensively using \texttt{QuickCheck} and several examples of 
both well-typed and ill-typed t-diagrams are present in the directory \texttt{examples}.

\subsection{Attribute Grammar}
In the attribute grammar file (\texttt{src/Type/AG.ag}) the abstract syntax tree containing
\texttt{Diag} and \texttt{Diag\_} is inspected and typed.
The type rules defined in \ref{sec:Spec} have been extended in order to smartly raise type-error
only when needed. For this purpose an ad-hoc type \texttt{ErrorT} has been added to the set
 of supported types. This type is produced when a t-diagram is not well-typed and when
a t-diagrams is composed with an ill-typed t-diagram.
Using this type avoids generating superflous type-error messages, but it still allows to collect
as many non-trivial error messages as possible.
The following rules formally specifies how this type is generated:

\sproof
\RightLabel{[ErrorT-base]}
\AxiomC{\blue{d} is ill-typed}
\UnaryInfC{\blue{d} : \red{ErrorT}}
\eproof

\sproof
\RightLabel{[ErrorT-compile]}
\AxiomC{\blue{d1} : \red{ErrorT} $\lor$  \blue{d2} : \red{ErrorT}}
\UnaryInfC{\blue{Compile d1 d2} : \red{ErrorT}}
\eproof

\sproof
\RightLabel{[ErrorT-execute]}
\AxiomC{\blue{d1} : \red{ErrorT} $\lor$  \blue{d2} : \red{ErrorT}}
\UnaryInfC{\blue{Execute d1 d2} : \red{ErrorT}}
\eproof
The attributes of the attribute grammar are:
\begin{itemize}
  \item \texttt{pos}
  \item \texttt{ty}
  \item \texttt{msgs}
\end{itemize}
The attribute \texttt{pos} is a \texttt{SourcePos} object simply passed around and used inside
error messages.
The attribute \texttt{ty} is the \texttt{Type} of a \texttt{Diag}.
The attribute \texttt{msgs} is a list collecting the error messages generated.

\section{Rendering}
This section describes the design and implementation of the program \texttt{tdiag2picutre}.

This program is split into two parts. The first stage inlines all variables, whereas the second stage then converts
a given diagram into a tree describing the latex commands to generate. The functionality is implemented in the
following modules:
\begin{itemize}
    \item \texttt{Diag2Picture.Inline}
    \item \texttt{Diag2Picture.Render}
\end{itemize}

\subsection{Inlining}
The inliner removes all \texttt{Let} and \texttt{VarAccess} nodes from a given diagram. Doing this can be
done in a straight-forward way by replacing all \texttt{VarAccess} by the diagram the variables refers to and
simply dropping all \texttt{Let} nodes afterwards. The implementation can be found in the file \texttt{src/Diag2Picture/Inline.ag}
and makes use of the following attributes:
\begin{itemize}
  \item \texttt{ndiag}
  \item \texttt{env}
  \item \texttt{repl}
\end{itemize}

The attribute \texttt{env} contains the environment. The attributes \texttt{ndiag} and \texttt{repl} contain the new built tree.
Because the \texttt{Decl} production returns a tuple of type \texttt{(String, Diag)}, the two attributes are required
to be able to substitute the variable value for the \texttt{VarAccess}, which has type \texttt{Diag\_}.

Please note that the behaviour of this module is undefined if the input diagram is ill-typed.

\subsection{Rendering}
\label{sec:Inlining}
The rendering component converts a diagram to a picture tree describing the latex commands to produce. The source code
of this module can be found in \texttt{src/Diag2Picture/Render.ag}.
The following attributes are used for rendering:
\begin{itemize}
    \item \texttt{pict}
    \item \texttt{size}
    \item \texttt{connStackTop}
    \item \texttt{connSide}
    \item \texttt{connStackBottom}
    \item \texttt{compConnSide}
    \item \texttt{dlng}
    \item \texttt{result}
    \item \texttt{result\_}
    \item \texttt{name}
    \item \texttt{nameOvr}
\end{itemize}
The attribute \texttt{pict} contains the rendered picture, the final output. The attribute \texttt{size} contains
the size of a given (sub-)tree.
The \texttt{conn...} nodes contain the coordinates where other components may be docked. Not all connection points
are defined for all elements, but all combinations which type-check are implemented. The \texttt{compConnSide} is a
special connection point, only defined for something which is of type \red{\texttt{Program \_ (Compiler \_ \_)}}. It is used to dock
elements to a compiler.

To produce the right hand side of a \texttt{Compile} node, the remaining attributes are used. The attribute
\texttt{dlng} contains the target language, and is only defined for nodes with a type of \red{\texttt{Program \_ (Compiler \_ \_)}}.
The attribute \texttt{result}/\texttt{result\_} contains a function which produces an element of the same type
as it's defining node with a given language. The expected language of the right hand side of a \texttt{Compile} is 
passed to this function, thereby producing the right element.
The two \texttt{results}/\texttt{result\_} are necessary to fix the type signatures, the same way as mentioned
in the section \ref{sec:Inlining}.
The attributes \texttt{name} and \texttt{nameOvr} are used to build the name of the right hande side of a \texttt{Compile}.
This is necessary because else the name of compiling an interpreter, executed on another interpreter, would not be known.
An example of this case can be seen in \texttt{examples/compile-exec.tdiag}. Further documentation of the
attributes is available directly in the source code.

Please note that some of the attributes are not well-defined for all productions. As we are not aware of any elegant
way to express this in an UUAG grammar, this leads to either some warnings or quite a bit of dummy code.

Joining together multiple elements is then simply moving them around so that the chosen connection points
of the individual elements end up at the same locations. As this might cause the diagram to extend to
negative coordinates, the joined diagram is shifted by the smallest found coordinates.
When the final positions have been found, all other position-dependent attributes
can be computed using the appropriate information from the incorporated elements and translating
it by the position of the elements.

All coordinates, line slopes etc. have been taken from the assignment and have not been changed.

The precondition of the rendering algorithm is, that the given diagram is well-typed and does not contain
any \texttt{Let} or \texttt{VarAccess} nodes. If this conditions are fullfilled, the rendering should
always succeed. 

\end{document}
